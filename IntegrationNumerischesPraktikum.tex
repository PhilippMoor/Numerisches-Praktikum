
\documentclass[12pt]{article}
\usepackage{graphicx}
\usepackage[T1]{fontenc}
\usepackage[utf8]{inputenc}
%\usepackage[ngerman]{babel}
\usepackage{amssymb}
\usepackage{amsmath}
\usepackage{siunitx}
\usepackage{lineno}
\usepackage{fancyhdr}

%commands
\newcommand{\monthwordlong}[1]{\ifcase#1 \or Januar\or Februar\or März\or April\or Mai\or Juni\or Juli\or August\or September\or Oktober\or November\or Dezember\fi}
\newcommand{\monthwordshort}[1]{\ifcase#1\or Jan\or Feb\or März\or Apr\or Mai\or Jun\or Jul\or Aug\or Sept\or Okt\or Nov\or Dez\fi} 
\newcommand{\datelong}{\number\day. \monthwordlong{\the\month} \number\year}
\newcommand{\dateshort}{\number\day. \monthwordshort{\the\month} \number\year}
\newcommand{\R}{\mathbb{R}}


\begin{document}
\pagestyle{fancy}
\fancyhf{}
\fancyhead[L]{\dateshort}
\fancyhead[R]{Philipp Moor, Jonas Guler}
\fancyfoot[R]{\thepage}

\begin{titlepage}
\clearpage
\title{Integration über eine beliebige Dreiecksfläche}
\author{Jonas Guler, Philipp Moor}
\date{\datelong}
\maketitle
\thispagestyle{empty}
\tableofcontents
\end{titlepage}

\newpage
%
% Start line numbering here if you want
%

%% main text
\section{Ausgangslage}
Aufgrund der anhaltenden Klimaerwärmung kann man seit mehreren Jahren einen Rückgang der Gletscher beobachten. Lange Zeit wurde der Schwund der Gletscher nur abgeschätzt. Da dies zwischen den Forschenden in sehr hohen Differenzen endete, wollte man die Volumen genauer berechnen können. Folgender Ansatz wurde verfolgt:\\
Zuerst wurde die gesamte Gletscheroberfläche in Dreiecke aufgeteilt. An allen Eckpunkten dieser Dreiecke wird mittels Schallwellen die Tiefe bis zum Gestein gemessen. Weiter wird anhand konkreter Funktionen die Unterseite der Gletscher modelliert um damit das Gesamtvolumen mit Hilfe eines zweidimensionalen Integral über die triangulierte Oberfläche berechnet. \\ \\ \\
Betrachten wir nun die Ausgangslage dieses Problems mathematisch: \\
Gegeben sei ein beliebiges Dreieck $\tau\subset\R^3$ und die Funktion $f$:
\[
\tau(i,j,k) :=
\left \{
		 \begin{pmatrix} i_1\\ i_2 \\ i_3 \end{pmatrix}
		 ,
		 \begin{pmatrix} j_1\\ j_2 \\ j_3 \end{pmatrix}
		 ,
		 \begin{pmatrix} k_1\\ k_2 \\ k_3 \end{pmatrix}
\right \}
\]
\[
   f(x) : \R^2 \rightarrow \R
\]
\\
Das Gletschervolumen $V$ über diesem Dreieck entspricht demnach folgenden Integral:
\[
V=\int_{\tau}f(x) dx
\]
Unser Ziel ist nun dieses Integral mit geeigneten Transformationen in ein Integral über dem Einheitsquadrat $Q:=[0,1]^2 \subset \R^2$ umzuformen, damit es in einen nächsten Schritt mit zweidimensionale Gauss - Quadratur approximiert werden kann.  \\
\\

\section[Transformation des Dreiecks ins Einheitsquadrat in $\R^2$]{Transformation des Dreiecks ins Einheits-\\quadrat in $\R^2$}
Zuerst gehen wir auf die Transformationen vom ursprünglichen Dreieck $\tau$ ins Einheitsquadrat $Q:=[0,1]^2$ ein.Um es ein bisschen übersichtlicher zu gestalten, nehmen wir uns ein Referenzdreieck $\tau_r\subset\R^2$ zur Hilfe. Gegeben sind also folgende Dreiecke
\[
\tau := \left \{ A,B,C \right \}
\quad\text{und}\quad
\tau_r := \left \{ \begin{pmatrix} 0\\0 \end{pmatrix},
\begin{pmatrix} 1\\0\end{pmatrix},
\begin{pmatrix} 1\\1\end{pmatrix} \right \}.
\]
Wir suchen also in einem ersten Schritt eine affine Funktion $\chi_{(i,j,k)}:\tau_r\rightarrow\tau$ so dass $\chi_{(i,j,k)}(\tau_r)=\tau$. Die Funktion ist definiert als
\[
\chi_{(i,j,k)}(\hat{x})= i + M\cdot \hat{x}, \quad M := \left[ \begin{pmatrix} j-i \end{pmatrix} |
\begin{pmatrix} k-j \end{pmatrix} \right] \in Mat(3\times 2,\R)
\]
wobei $\hat{x}\in \tau_r$ und $(i,j,k)\in\sigma(A,B,C)$.\\
Bemerke, die Indizes $(i,j,k)$ definieren wie die Eckpunkte von $\tau$ auf die Eckpunkte von $\tau_r$ abgebildet werden. Durch eine geeignete Wahl der Permutation können wir die Transformation beeinflussen, was sich später auf Konvergenzgeschwindigkeit der Gauss - Quadratur auswirkt.
In einem zweiten Schritt transformieren wir das Referenzdreieck ins Einheitsquadrat $Q$. Dafür definieren wir folgende Funktion:
\[
\rho : Q \rightarrow \tau_r,\quad \begin{pmatrix} \mu,\nu \end{pmatrix} \mapsto
\begin{pmatrix} \mu\\\mu\cdot\nu \end{pmatrix}
\]
Betrachten wir nun das ursprüngliche Integral und wenden den Transformationssatz doppelt mit den obigen Funktionen an.
\begin{align*}
V &=\int_{\tau}f(x) dx\\
&= \int_{\tau_r}(f\circ\chi_{(i,j,k)})(\hat{x}) \cdot \sqrt{det(M^T\cdot M)} dx\\
&=\int_{Q} (f\circ\chi_{(i,j,k)}\circ\rho)(\mu,\nu) \cdot \sqrt{det(M^T\cdot M)} \cdot \sqrt{det(J_\rho)} dx\\
&=\int_{0}^{1}\int_{0}^{1} 
\end{align*}
\\

\newpage

\section{Behandelte Funktionen und ihre Problemstellen}


Probleme die mit der Transformation auftauchen --> Singularitäten und ihre Lösungen

\newpage

\section{Qauss - Quadratur}
\subsection{Berechnung der Stützstellen und Gewichte}

\newpage

\section{Tests}

Für das Testen unserer Methoden spielen verschiedene Parameter eine entscheidende Rolle, daher bauten wir eine Test-Umgebung in Matlab, die es uns ermöglichte diese Parameter zu verändern und Plots zu erstellen um Auswertungen vorzunehmen.
\\
Zu den Parametern die Einfluss auf das Resultat haben können gehören :

\begin{itemize}
	\item Form des Referenzdreiecks \& Positionierung im $\R^3$
	\item Gutartigkeit der Funktion bzgl. Singularitäten
	\item Abhängigkeit von der Wahl des Referenzeckpunktes
	\item Anzahl der Gauss - Stützstellen
\end{itemize}

Unser Auftrag war es konkret unsere Methode an folgenden Funktionen unter beliebiger Wahl des Referenzdreiecks zu Testen:




\begin{itemize}
	\item Polynome beliebigen Grades
	\item $f(y) = \frac{1}{||y - A_l||} $ mit $A_t$ einem fixierten Eckpunkt des Dreiecks
	\item $g(y) = log(||x - y||)  , \text{mit} \quad x = (x_1,x_2,x_3) \in \R^3\quad$ fixiert 
	\item $h(y) =x\cdot log(x) $
\end{itemize}

Die Wahl der Test-Funktionen wurde so getroffen, um die Methode gezielt auf kritische Resultate oder eine allfällige Run-Time-Inefficency zu untersuchen.\\
Bei den Polynomen liegt der Fokus darauf, die Präzision der Methode unter wachsendem Grad des Polynoms zu untersuchen.
\\
Bei $f(y)$ liegt die Problematik innerhalb des Betrags im Nenner des Bruchs. Wenn sich der Wert $y$ dem Eckpunkt des Dreiecks nähert, kann abhängig der Wahl des Referenzeckpunktes eine Singularität auftreten.\\
Die Funktion $g(y)$ bewegt sich sehr schnell gegen $-\infty$ wenn $x$ und $y$ sich einander annähern, was insbesondere zu einer starken Verfälschung des Resultats führen kann.\\


\subsection{Testverfahren}

Das Testen dieser Funktionen


\newpage

\section{Resultate}

\newpage



\end{document}