
\documentclass[12pt]{article}
\usepackage{graphicx}
\usepackage{amssymb}
\usepackage{amsmath}
\usepackage{siunitx}
\usepackage{lineno}
%\journal{Journal of Mathematical Stupidity}
\begin{document}

%\begin{frontmatter}
\title{Polynomials with infinite solutions}
\author{Jonas Guler, Philipp Moor}
\maketitle
%\end{frontmatter}

%%
%% Start line numbering here if you want
%%

%% main text
\section{Die Transformation des Dreiecks}
Ausgangslage des Integrationsproblems ist ein Dreieck im $\mathbb{R}^3$, und eine Funktion $f$:
\[
\tau(i,j,k) =
\left \{
		 \begin{pmatrix} i_1\\ i_2 \\ i_3 \end{pmatrix}
		 ,
		 \begin{pmatrix} j_1\\ j_2 \\ j_3 \end{pmatrix}
		 ,
		 \begin{pmatrix} k_1\\ k_2 \\ k_3 \end{pmatrix}
\right \}
\]


\[
   f(x) : \mathbb{R}^2 \rightarrow \mathbb{R}
\]

Um die zweidimensionale Gauss-Quadratur anzuwenden, muss das Dreieck affin auf das Einheitsquadrat in $\mathbb{R}^2$ abgebildet werden.

\section{Transformation des Dreiecks ins $\\$ Einheitsquadrat in $\mathbb{R}^2$}
Sei $\tau\subset \mathbb{R}^3$ ein beliebiges Dreieck (A,B,C)
\[
\tau = \left \{
\begin{pmatrix} a_1\\ a_2 \\ a_3 \end{pmatrix}
,
\begin{pmatrix} b_1\\ b_2 \\ b_3 \end{pmatrix}
,
\begin{pmatrix} c_1\\ c_2 \\ c_3 \end{pmatrix}
\right \}
\]

und $\tau_r \subset \mathbb{R}^2$ ein Referenzdreieck mit
\[
 \tau_r = \left \{
\begin{pmatrix} 0\\ 0 \end{pmatrix}
,
\begin{pmatrix} 1\\ 0\end{pmatrix}
,
\begin{pmatrix} 1\\ 1\end{pmatrix}
\right \} .
\]

Gesucht wird eine affine Funktion $\chi_i$ von $\tau$ nach $\tau_r$ so dass: $\chi(\tau) = \tau_r$

Die Funktion ist gegeben durch:
\[
\chi_I(x) = I + M\cdot x, \quad M := \left[
\begin{pmatrix} B-A \end{pmatrix}
|
\begin{pmatrix} B-C \end{pmatrix}
\right]
 \in Mat(3\times 2,\mathbb{R})
\]
wobei I der Eckpunkt von $\tau$ ist, der auf 0 abgebildet wird.

Ausgehend von diesem Referenzdreieck wird die Transformation ins Einheitsquadrat sehr einfach. Sie ist durch folgende Funktion gegeben:

\[
	\rho : \mathbb{R}^2 \rightarrow \mathbb{R}^2, \begin{pmatrix} \mu,\nu \end{pmatrix} \mapsto \begin{pmatrix} \mu\\\mu\cdot\nu \end{pmatrix}
\]


 

\newpage


\section{A polynomial of degree infinity}

Most of the time when we consider polynomials, we consider the degree to be finite.
However, there are instances when a polynomial can be thought of as having an infinite degree. Like a power series.
\[
    \sum_{n = 0}^{\infty} \frac{f^{(n)} (a)}{n!} (x-a)^n
\]
A Taylor series is defined to be infinite.
It is called a Maclaurin series when $a = 0$.
\label{S:2}
\section{Roots of a degree infinity polynomial}
By the fundamental theorem of algebra, we know that a degree $n$ polynomial's equation has $n$ roots.
Therefore, a polynomial of degree $\infty$ has $\infty$ solutions.

Given a polynomial $P(x)$, let us assume that it has a solution set $S$.
Let us also assume that the solution set is finite, i.e., there are a finite number of solutions for $P(x)$.
Now, let us take $n$ to be the largest root in $S$.
Since it is a root, 
\[
    P(n) = 0
\]
\begin{equation} \label{n_root}
     a_0 + a_1n + a_2n^2 + ... = 0
\end{equation}
   
Let us also take any other arbitrary $m$.
Now, let us see if $m + n$ is a root of $P(x)$.
\begin{align*}
    a_0 + a_1(n + m) + a_2(n + m)^2 + ... &= 0\\
    a_0 + a_1n + a_1m + a_2(n^2 + 2mn + m^2) + ... &= 0\\
    a_0 + a_1n + a_1m + a_2n^2 + a_2m^2 + 2a_2mn + ... &= 0
\end{align*}
We see that simplifying produces $a_in^i + a_im^i$ along with all the intermediate terms of a bionomial expansion of the form $(a \pm b)^n$.
Let the sum of all these intermediate terms be $T_{m + n}$.

Now, 
\begin{align*}
    a_1n + a_1m + a_2n^2 + a_2m^2 + ... + T_{m+n} &= 0\\
    (a_1n + a_2n^2 + ...) + (a_1m + a_2m^2 + ...) + T_{m+n} &= 0
\end{align*}
But from \ref{n_root}, 
\begin{equation} \label{mn_root}
    (a_1m + a_2m^2 + ...) + T_{m+n} = 0
\end{equation}
Since it is possible for the expression \ref{mn_root}, 
\[P(m+n)=0\]
But $n$ is by definition, the largest root, then, by contradiction, the solution set $S$ has to be infinite.
Therefore, in this case,there are infinite solutions to the polynomial $P(x)$.

But what if there is no value of $m$ that satisfies $T_{m+n}=0$? Then, in that case, the solution set can be thought of as not having as many elements as the degree of the polynomial.
Just like how 
\[
    x^2 + 1 - 2x = 0
\]
Has one solution $x=1$.
The solution set for this quadratic would be $\{1\}$.
Here, there are two roots - but they are both equal. The case when $m$ does not exist would also be similar.
In other words, the infinity of the solution set ($\infty_{S}$) would be smaller than or equal to the infinity of the degree of the polynomial ($\infty_{P}$), for any given infinite polynomial.
Or,
\begin{equation}
    \infty_{S} \leqslant \infty_{P}
\end{equation}
\section{Such equations in action}
Here are a few examples :
\[
\sum_{n=0}^{\infty} \dfrac{(-1)^n}{(2n+1)!} x^{2n+1}
\]
This is the Taylor series for the $sin(x)$ function. It is how your calculator get the value when you feed $x$ to it.
\section*{A note on other fields}
A field is a set on which the binary operations $+$, $-$, $\times$ and $\div$ are defined.
Besides integral fields, there are other fields where polynomials will behave differently and a polynomial with finite terms and of a finite degree can also have infinite solutions.
%% The Appendices part is started with the command \appendix;
%% appendix sections are then done as normal sections
%% \appendix

%% \section{}
%% \label{}

%% References
%%
%% Following citation commands can be used in the body text:
%% Usage of \cite is as follows:
%%   \cite{key}          ==>>  [#]
%%   \cite[chap. 2]{key} ==>>  [#, chap. 2]
%%   \citet{key}         ==>>  Author [#]

%% References with bibTeX database:

\bibliographystyle{model1-num-names}
\appendix
\section*{Bibliography}
\bibliography{sample.bib}

%% Authors are advised to submit their bibtex database files. They are
%% requested to list a bibtex style file in the manuscript if they do
%% not want to use model1-num-names.bst.

%% References without bibTeX database:

% \begin{thebibliography}{00}

%% \bibitem must have the following form:
%%   \bibitem{key}...
%%

% \bibitem{}

% \end{thebibliography}


\end{document}

%%
%% End of file `elsarticle-template-1-num.tex'.