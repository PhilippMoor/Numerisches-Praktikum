
\documentclass[preprint,12pt]{elsarticle}
\usepackage{graphicx}
\usepackage{amssymb}
\usepackage{amsmath}
\usepackage{siunitx}
\usepackage{lineno}
\journal{Journal of Mathematical Stupidity}
\begin{document}
\begin{frontmatter}
\title{Polynomials with infinite solutions}
\author{Rahul Chhabra}
\address{St. Joseph's College, Prayagraj, Uttar Pradesh, India}
\begin{abstract}
A solution to any polynomial $P(x)$ is a value of $x$ that satisfies $P(x) = 0$ . A polynomial of degree 1 forms an equation called a "linear" equation. A linear equation can be expressed as $ax + b = 0$. Its solution is then : 
\[
    x = \frac{-b}{a}
\]
For polynomials of degree 2, things go a little complicated :
\[
    P(x) = ax^2 + bx + c
\]
Nonetheless, a quadratic equation can be solved rather easily with the help of the quadratic formula :
\[
    x = \frac{-b \pm \sqrt{b^2 - 4ac}}{2a} 
\]
Cubics and quartics each have 3 and 4 roots respectively.
Overall, any polynomial of degree $n$ has exactly $n$ solutions or roots. 
But can there be a polynomial equation which has infinite solutions?
As it turns out - yes.
\end{abstract}

\begin{keyword}
Polynomial \sep solution \sep degree of a polynomial \sep linear 
\sep quadratic \sep quadratic formula \sep roots \sep cubics \sep quartics
%% keywords here, in the form: keyword \sep keyword

%% MSC codes here, in the form: \MSC code \sep code
%% or \MSC[2008] code \sep code (2000 is the default)

\end{keyword}

\end{frontmatter}

%%
%% Start line numbering here if you want
%%

%% main text
\section{The null polynomial}
In general, any polynomial can be expressed as :
\[
    P(x) = a_0 + a_1x + a_2x^2 + ...  + a_ix^i
\]
The null polynomial is a polynomial that returns 0 for any value of $x$.
$P(x) = 0$
It can be understood better as a polynomial with every coefficient equal to 0:
\[
    P(x) = 0 + 0x + 0x^2 + ... + 0x^i
\]
It will return 0 no matter what. Since it returns 0 for any given value of $x$, it has infinite solutions.\cite{1137202}
\label{S:1}

\section{A polynomial of degree infinity}

Most of the time when we consider polynomials, we consider the degree to be finite.
However, there are instances when a polynomial can be thought of as having an infinite degree. Like a power series.
One prime example is of a Taylor series.\cite{1137203}
\[
    \sum_{n = 0}^{\infty} \frac{f^{(n)} (a)}{n!} (x-a)^n
\]
A Taylor series is defined to be infinite.
It is called a Maclaurin series when $a = 0$.
\label{S:2}
\section{Roots of a degree infinity polynomial}
By the fundamental theorem of algebra, we know that a degree $n$ polynomial's equation has $n$ roots.
Therefore, a polynomial of degree $\infty$ has $\infty$ solutions.

Given a polynomial $P(x)$, let us assume that it has a solution set $S$.
Let us also assume that the solution set is finite, i.e., there are a finite number of solutions for $P(x)$.
Now, let us take $n$ to be the largest root in $S$.
Since it is a root, 
\[
    P(n) = 0
\]
\begin{equation} \label{n_root}
     a_0 + a_1n + a_2n^2 + ... = 0
\end{equation}
   
Let us also take any other arbitrary $m$.
Now, let us see if $m + n$ is a root of $P(x)$.
\begin{align*}
    a_0 + a_1(n + m) + a_2(n + m)^2 + ... &= 0\\
    a_0 + a_1n + a_1m + a_2(n^2 + 2mn + m^2) + ... &= 0\\
    a_0 + a_1n + a_1m + a_2n^2 + a_2m^2 + 2a_2mn + ... &= 0
\end{align*}
We see that simplifying produces $a_in^i + a_im^i$ along with all the intermediate terms of a bionomial expansion of the form $(a \pm b)^n$.
Let the sum of all these intermediate terms be $T_{m + n}$.

Now, 
\begin{align*}
    a_1n + a_1m + a_2n^2 + a_2m^2 + ... + T_{m+n} &= 0\\
    (a_1n + a_2n^2 + ...) + (a_1m + a_2m^2 + ...) + T_{m+n} &= 0
\end{align*}
But from \ref{n_root}, 
\begin{equation} \label{mn_root}
    (a_1m + a_2m^2 + ...) + T_{m+n} = 0
\end{equation}
Since it is possible for the expression \ref{mn_root}, 
\[P(m+n)=0\]
But $n$ is by definition, the largest root, then, by contradiction, the solution set $S$ has to be infinite.
Therefore, in this case,there are infinite solutions to the polynomial $P(x)$.

But what if there is no value of $m$ that satisfies $T_{m+n}=0$? Then, in that case, the solution set can be thought of as not having as many elements as the degree of the polynomial.
Just like how 
\[
    x^2 + 1 - 2x = 0
\]
Has one solution $x=1$.
The solution set for this quadratic would be $\{1\}$.
Here, there are two roots - but they are both equal. The case when $m$ does not exist would also be similar.
In other words, the infinity of the solution set ($\infty_{S}$) would be smaller than or equal to the infinity of the degree of the polynomial ($\infty_{P}$), for any given infinite polynomial.
Or,
\begin{equation}
    \infty_{S} \leqslant \infty_{P}
\end{equation}
\section{Such equations in action}
Here are a few examples :
\[
\sum_{n=0}^{\infty} \dfrac{(-1)^n}{(2n+1)!} x^{2n+1}
\]
This is the Taylor series for the $sin(x)$ function. It is how your calculator get the value when you feed $x$ to it.
\section*{A note on other fields}
A field is a set on which the binary operations $+$, $-$, $\times$ and $\div$ are defined.
Besides integral fields, there are other fields where polynomials will behave differently and a polynomial with finite terms and of a finite degree can also have infinite solutions.
%% The Appendices part is started with the command \appendix;
%% appendix sections are then done as normal sections
%% \appendix

%% \section{}
%% \label{}

%% References
%%
%% Following citation commands can be used in the body text:
%% Usage of \cite is as follows:
%%   \cite{key}          ==>>  [#]
%%   \cite[chap. 2]{key} ==>>  [#, chap. 2]
%%   \citet{key}         ==>>  Author [#]

%% References with bibTeX database:

\bibliographystyle{model1-num-names}
\appendix
\section*{Bibliography}
\bibliography{sample.bib}

%% Authors are advised to submit their bibtex database files. They are
%% requested to list a bibtex style file in the manuscript if they do
%% not want to use model1-num-names.bst.

%% References without bibTeX database:

% \begin{thebibliography}{00}

%% \bibitem must have the following form:
%%   \bibitem{key}...
%%

% \bibitem{}

% \end{thebibliography}


\end{document}

%%
%% End of file `elsarticle-template-1-num.tex'.